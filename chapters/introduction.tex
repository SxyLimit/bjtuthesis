\cleardoublepage{}

\chapter{绪论}
引言(或绪论)简要说明研究工作的目的、范围、相关领域的前人工作和知识空白、理论基础和分析、研究设想、研究方法和实验设计、预期结果和意义等。应言简意赅,不要与摘要雷同,不要成为摘要的注释。一般教科书中有的知识,在引言中不必赘述。

\section{编译方式说明}

\subsection*{在 Overleaf 上编译}

请注意以下事项:

\begin{itemize}
  \item 请上传整个 \verb|.zip| 压缩包。
  \item 不要删除 \verb|.latexmkrc| 文件,Overleaf 会自动识别并使用 XeLaTeX 编译(你也可以手动设置为 XeLaTeX)。
  \item 在 \verb|chapter| 文件夹下编辑你的各章节内容。
\end{itemize}

\subsection*{在本地编译}

请按照以下步骤操作:

\begin{itemize}
  \item 打开主 \verb|.tex| 文件,例如:\verb|bjtuthesis.tex|。
  \item 进入:选项 $\rightarrow$ 设置 TexStudio $\rightarrow$ 构建,设置如下:
    \begin{itemize}
      \item 默认编译器选择:\verb|XeLaTeX|
      \item 默认文献工具选择:\verb|Biber|
    \end{itemize}
  \item 点击工具栏绿色箭头“构建 \& 查看”(或按 \verb|F5|)。
  \item 如果文献未显示,点击顶部菜单“工具(T)” $\rightarrow$ “参考文献(B)”运行 Biber(或按 \verb|F8|)。
  \item 然后运行 XeLaTeX 两次,即点击两次“构建 \& 查看”(\verb|F5|)。
\end{itemize}

\vspace{1em}
\noindent 快捷方式:设置完成后,按顺序使用快捷键 \verb|F5 - F8 - F5 - F5| 即可完整编译。

\subsection*{在服务器(Linux)编译}

建议使用 \verb|Makefile|,集成了编译、清理中间文件、打开 PDF 三个步骤。

注意:使用 \verb|Makefile| 时请勿修改主 \verb|.tex| 文件名。

可用命令如下:

\begin{itemize}
  \item \verb|make|:等同于 \verb|make all|,执行清理后编译 PDF。
  \item \verb|make bjtuthesis|:仅编译 PDF。
  \item \verb|make clean|:仅清理中间文件。
  \item \verb|make view|:仅打开生成的 PDF。
\end{itemize}


\section{统计字数脚本}

模板提供了统计字数脚本,可在服务器或本地运行:

\subsection*{使用 Bash 运行 \texttt{word\_count.sh}}

\begin{enumerate}
  \item 打开终端(Windows 推荐使用 Git Bash 或 WSL + VS Code),切换到项目目录:
  
  \begin{verbatim}
cd path\to\your\project
  \end{verbatim}

  \item 执行脚本,传入主 \verb|.tex| 文件名:
  
  \begin{verbatim}
./scripts/word_count.sh bjtuthesis.tex
  \end{verbatim}

  \item 输出说明:
  \begin{itemize}
    \item 控制台显示简要统计信息;
    \item 完整统计信息输出至:\verb|outputs/bjtuthesis.wordcnt|。
  \end{itemize}
\end{enumerate}

\subsection*{使用 PowerShell 运行 \texttt{word\_count.ps1}}

\begin{enumerate}
  \item 打开终端,切换到项目目录:

  \begin{verbatim}
cd path\to\your\project
  \end{verbatim}

  \item 执行脚本,传入主 \verb|.tex| 文件名:

  \begin{verbatim}
./scripts/word_count.ps1 bjtuthesis.tex
  \end{verbatim}

  \item 输出说明:
  \begin{itemize}
    \item 控制台显示简要统计信息;
    \item 完整统计信息输出至:\verb|outputs/bjtuthesis.wordcnt|。
  \end{itemize}
\end{enumerate}

